%%%%%%%%%%%%%%%%%%%%%%%%%%%%%%%%%%%%%%%%%
% Beamer Presentation
% LaTeX Template
% Version 1.0 (10/11/12)
%
% This template has been downloaded from:
% http://www.LaTeXTemplates.com
%
% License:
% CC BY-NC-SA 3.0 (http://creativecommons.org/licenses/by-nc-sa/3.0/)
%
%%%%%%%%%%%%%%%%%%%%%%%%%%%%%%%%%%%%%%%%%

%----------------------------------------------------------------------------------------
%	PACKAGES AND THEMES
%----------------------------------------------------------------------------------------

\documentclass{beamer}

\mode<presentation> {

% The Beamer class comes with a number of default slide themes
% which change the colors and layouts of slides. Below this is a list
% of all the themes, uncomment each in turn to see what they look like.

%\usetheme{default}
%\usetheme{AnnArbor}
%\usetheme{Antibes}
%\usetheme{Bergen}
%\usetheme{Berkeley}
%\usetheme{Berlin}
%\usetheme{Boadilla}
%\usetheme{CambridgeUS}
%\usetheme{Copenhagen}
%\usetheme{Darmstadt}
%\usetheme{Dresden}
%\usetheme{Frankfurt}
%\usetheme{Goettingen}
%\usetheme{Hannover}
%\usetheme{Ilmenau}
%\usetheme{JuanLesPins}
%\usetheme{Luebeck}
\usetheme{Madrid}
%\usetheme{Malmoe}
%\usetheme{Marburg}
%\usetheme{Montpellier}
%\usetheme{PaloAlto}
%\usetheme{Pittsburgh}
%\usetheme{Rochester}
%\usetheme{Singapore}
%\usetheme{Szeged}
%\usetheme{Warsaw}

% As well as themes, the Beamer class has a number of color themes
% for any slide theme. Uncomment each of these in turn to see how it
% changes the colors of your current slide theme.

%\usecolortheme{albatross}
%\usecolortheme{beaver}
%\usecolortheme{beetle}
%\usecolortheme{crane}
%\usecolortheme{dolphin}
%\usecolortheme{dove}
%\usecolortheme{fly}
%\usecolortheme{lily}
%\usecolortheme{orchid}
%\usecolortheme{rose}
%\usecolortheme{seagull}
%\usecolortheme{seahorse}
%usecolortheme{whale}
%\usecolortheme{wolverine}

%\setbeamertemplate{footline} % To remove the footer line in all slides uncomment this line
%\setbeamertemplate{footline}[page number] % To replace the footer line in all slides with a simple slide count uncomment this line

%\setbeamertemplate{navigation symbols}{} % To remove the navigation symbols from the bottom of all slides uncomment this line
}

\usepackage{graphicx} % Allows including images
\usepackage{booktabs} % Allows the use of \toprule, \midrule and \bottomrule in tables
\usepackage{amsmath}
%----------------------------------------------------------------------------------------
%	TITLE PAGE
%----------------------------------------------------------------------------------------

\title[LLM]{Lifetime Limited Memory} % The short title appears at the bottom of every slide, the full title is only on the title page

\author{Grant Baker, Matt Maierhofer} % Your name
\institute[CU] % Your institution as it will appear on the bottom of every slide, may be shorthand to save space
{
University of Colorado \\ % Your institution for the title page
\medskip
\textit{} % Your email address
}
\date{\today} % Date, can be changed to a custom date

\begin{document}

\begin{frame}
\titlepage % Print the title page as the first slide
\end{frame}

\begin{frame}
\frametitle{Overview} % Table of contents slide, comment this block out to remove it
\tableofcontents % Throughout your presentation, if you choose to use \section{} and \subsection{} commands, these will automatically be printed on this slide as an overview of your presentation
\end{frame}

%----------------------------------------------------------------------------------------
%	PRESENTATION SLIDES
%----------------------------------------------------------------------------------------

%------------------------------------------------
\section{Support Vector Clustering} % Sections can be created in order to organize your presentation into discrete blocks, all sections and subsections are automatically printed in the table of contents as an overview of the talk
%------------------------------------------------

%\subsection{Subsection Example} % A subsection can be created just before a set of slides with a common theme to further break down your presentation into chunks

\begin{frame}
\frametitle{Support Vector Clustering}
\begin{itemize}
\item Introduced by Ben-Hur et al. \cite{SVC}
\item Projects data to infinite dimensional space
\item Finds the smallest bounding hypersphere on the data, allowing penalized outliers
\item Sphere is projected back to the data space, forming clusters on the data
\end{itemize}
\end{frame}

%------------------------------------------------

\begin{frame}
\frametitle{The Primal Problem}
\begin{align} \label{P1}
    \min_{R,a,\xi} \textit{ } &R^2 + C\sum_{j} \xi_j\nonumber \\
    &||\phi(X_i) - a||^2\leq R^2 + \xi_i\nonumber\\
    &\xi_i \geq 0 \textit{ }\nonumber\\~ \nonumber\\
    &X\in \mathbb{R}^{m\times n}\nonumber, \;\xi \in \mathbb{R}^n
\end{align}

\end{frame}
\begin{frame}{The Lagrangian}
    \begin{align*}
    \mathcal{L}(R,a,\xi,\beta,\mu) =& R^2 + C\sum_{j=1}^n \xi_j  \\&-\sum_{j=1}^n\beta_j( R^2 + \xi_j-||\phi(X_j) - a||^2) - \sum_{j=1}^n \xi_j  \mu_j
\end{align*}
\end{frame}
\begin{frame}{KKT Conditions - Stationarity}
    \begin{align*}
    &\nabla \mathcal{L}=0  \\\underline{\hspace{1cm}}&\underline{\hspace{3cm}}\\
    &\sum_{j=1}^n \beta_j = 1
    \\ &a = \sum_{j=1}^n \beta_j \phi(X_j)
    \\ &\beta_j = C - \mu_j
\end{align*}
\end{frame}
\begin{frame}{KKT Conditions - Complementary Slackness}
    \begin{align*}
    \beta_j( R^2 + \xi_j-||\phi(X_j) - a||^2) &= 0\nonumber\\
    \xi_j  \mu_j &= 0
\end{align*}
\end{frame}

\begin{frame}{The Dual Problem}
    \begin{align*}
    \max_{\mu,\beta,R,a,\xi} \mathcal{L}(R,a,\xi,\beta,\mu) \\
    \nabla_{R,a,\xi}\mathcal{L}= 0 \nonumber
\end{align*}
\end{frame}

\begin{frame}{Using Stationarity}
    Let $\hat{L}(\beta) = \mathcal{L}$ under our stationarity conditions.

\begin{align*}
    \hat{L}(\beta)&= R^2 + C\sum_{j} \xi_j - \sum_j\beta_j( R^2 + \xi_j-||\phi(X_j) - a||^2) - \sum_{j} \xi_j  \mu_j \nonumber\\
    &= R^2 (1-\sum_j\beta_j) + \sum_{j} \xi_j(C-\beta_j-\mu_j) + \sum_j\beta_j(||\phi(X_j) - a||^2)\nonumber\\
    &=\sum_j\beta_j||\phi(X_j)- a||^2 \nonumber\\
    &=\sum_j \beta_j(||\phi(X_j)||^2 - 2 a \cdot \phi(X_j) + ||a||^2)\nonumber\\
    &= \sum_j \beta_j\Big(||\phi(X_j)||^2 - 2 \sum_i\beta_i\phi(X_i) \cdot \phi(X_j)\Big) + ||\sum_i\beta_i\phi(X_i)||^2 \nonumber\\
    & = \sum_j\beta_j \phi(X_j)\cdot \phi(X_j)  - \sum_{i,j}\beta_i\beta_j \phi(X_i)\cdot \phi(X_j) 
\end{align*}
\end{frame}
\begin{frame}{What is $\phi$?}
    \begin{itemize}
        \item $\phi(x)$ maps $x$ from an $n$-dimensional data space to an infinite dimensional feature space
        \item We don't actually ever compute $\phi(x)$, instead we use the ``Kernel Trick''
        \item Define $K(x_1,x_2) = \phi(x_1)\cdot \phi(x_2)= e^{-q||x_1-x_2||_2^2}$
        \item Note $K(x,x) = 1$
    \end{itemize}
\end{frame}
\begin{frame}{The Dual - Reworked}
    \begin{equation*}
    \label{eq:dual}
    \max_{\beta\geq 0} \left(1 - \sum_{i,j}\beta_i\beta_j K(X_i,X_j)\right)
\end{equation*}
\end{frame}
\begin{frame}{Defining the Radius}
Note that if $0<\beta_j<C$, $\phi(X_j)$ is on the surface of the hypersphere, and we define \[R^2(x)=||\phi(x)-a||^2=K(x,x) - 2 \sum_j \beta_j K(x,X_j)+\sum_{i,j}\beta_i \beta_j K(X_i,X_j)\]as the distance from the center of the sphere, $R = R(X_j)$ is the radius of our hypersphere.
\end{frame}
\begin{frame}{Clustering}
Finally, we build an adjacency matrix for the points. Two points, $(X_i,X_j)$, are considered adjacent if the line between them in the data space is within the hypersphere in the feature space, or \[R(x) \leq R \;\;\forall x\in\{X_i + t(X_j-X_i)|t\in[0,1]\}\]
In practice, we sample points on the line to check. After we build the adjacency matrix, clusters of points are determined via a breadth first search of the matrix. 
\end{frame}
\section{Fast Support Vector Clustering}
\begin{frame}{Speeding Up the Adjacency Matrix}
    \begin{itemize}
        \item Computing the Adjacency Matrix can be very costly, $O(mn^3)$, want to reduce overall cost
        \item Ideally want to find a subset of ``important'' points and cluster using these
        \item Fast Support Vector Clustering\cite{Pham2017}, introduced by Pham et al., helps solve this
    \end{itemize}
\end{frame}
\begin{frame}{Fast Support Vector Clustering}
\begin{itemize}
    \item First, take the set of all points within $\epsilon$ of the boundary. This should give a solid representation of the boundary
    \item Next, we use a fixed point iterator to push our points to an equilibrium at a local optima of our radius function. \[x_{k+1} = \frac{\sum_j \beta_j e^{-q||X_j - x_k||^2} X_j}{\sum_j \beta_j e^{-q||X_j - x_k||^2}}\]
    \item We then cluster these ``equilibrium points'' using the adjacency matrix method
    \item The boundary points are assigned to the same cluster as the corresponding equilibrium points, and the other points are assigned to the same cluster as the closest boundary point in the data space.
\end{itemize}
\end{frame}
\section{Scalable Support Vector Clustering}
\begin{frame}{Speeding Up the Training Process}
    \begin{itemize}
        \item Scalable Support Vector Clustering, Pham et al. \cite{ssvc}
        \item Instead of solving a quadratic program, we fit the primal problem into the stochastic gradient framework
        \item Introduce a budget for the maximum number of points with nonzero weight
    \end{itemize}
\end{frame}
\begin{frame}{Scalable Support Vector Clustering}
    \begin{enumerate}
        \item Draw $X_i$ at random from $X$
        \item Update the weights $\beta_j \leftarrow \frac{t-1}{t} \beta_j$
        \item Check if $X_i$ is outside the cluster, i.e. if
        \[
            f(X_i) = 2 \sum_j \beta_j K(X_j, X_i) - R < 0
        \]
        \item If so, update $\beta_i \leftarrow \beta_i + 2 C R \eta_t$
        \item If there are more than B nonzero $\beta_j$, set the smallest one to $0$
    \end{enumerate}
\end{frame}
\section{Results}
\begin{frame}{Two Moons}
\includegraphics[width=\textwidth]{tm.png}
\end{frame}
\begin{frame}{KMeans}
\includegraphics[width=\textwidth]{kmeans.png}
\end{frame}
\begin{frame}{Two Moons with SVC}
\includegraphics[width=\textwidth]{img5.png}
\end{frame}
\begin{frame}{Two Moons Failure}
\includegraphics[width=\textwidth]{img6.png}
\end{frame}
\begin{frame}{Two Moons Failure}
\includegraphics[width=\textwidth]{img1.png}
\end{frame}
\begin{frame}{Two Moons Failure}
\includegraphics[width=\textwidth]{img2.png}
\end{frame}
\begin{frame}{Two Moons Failure}
\includegraphics[width=\textwidth]{wtf.png}
\end{frame}
\begin{frame}{Distorted Blobs}
\includegraphics[width=\textwidth]{img3.png}
\end{frame}
%------------------------------------------------

\begin{frame}
\frametitle{References}
\bibliographystyle{alpha}
\bibliography{sample}
\end{frame}

%------------------------------------------------

\begin{frame}
\begin{center}
\includegraphics[width=0.6\textwidth]{SkoBuffs.jpg}
\end{center}
\end{frame}

%----------------------------------------------------------------------------------------

\end{document}